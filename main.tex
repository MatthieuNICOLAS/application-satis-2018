\documentclass{article}

\usepackage[T1]{fontenc}
\usepackage{ucs}
\usepackage[utf8]{inputenc}

\usepackage{acronym} % \ac[p], \acl[p], \acs[p], \acf[p]
\usepackage{subcaption}
\captionsetup{subrefformat=parens}

\usepackage[inline]{enumitem}

\usepackage{graphicx}
\usepackage{color}
\AtBeginDocument{
\definecolor{pdfurlcolor}{rgb}{0,0,0}
\definecolor{pdfcitecolor}{rgb}{0,0,0}
\definecolor{pdflinkcolor}{rgb}{0,0,0}
\definecolor{light}{gray}{.85}
\definecolor{vlight}{gray}{.95}
\definecolor{darkgreen}{rgb}{0.0, 0.2, 0.13}
}
\usepackage[colorlinks=true,
            citecolor=pdfcitecolor,
            urlcolor=pdfurlcolor,
            linkcolor=pdflinkcolor,
            pdfborder={0 0 0}
            ]{hyperref}
\usepackage{url}
\urlstyle{rm}
\usepackage{paralist}
\usepackage{booktabs}

% Acronyms
% --------
\acrodef{CRDT}[CRDT]{Conflict-free Replicated Data type}
\acrodefplural{CRDT}[CRDTs]{Conflict-free Replicated Data types}

% Meta-Data
% ---------
\title{Efficient (re)naming in \acp{CRDT}}
\author{Matthieu Nicolas, Gérald Oster and Olivier Perrin}
% \affiliation{Inria, F-54600,
%   Université de Lorraine, LORIA, F-54506,
%   CNRS, F-54506}
% \email{Contact Author: matthieu.nicolas@inria.fr}

\hypersetup{pdftitle={Efficient (re)naming in CRDT},
  pdfsubject={},
  pdfkeywords={Data replication, CRDT},
  pdfauthor={M. Nicolas, O. Perrin, G. Oster}}

\pagestyle{empty}

\begin{document}

\maketitle
\thispagestyle{empty}

% Context
% - Distributed system: several nodes share and edit a data structure
% - CAP theorem: can not ensure strong consistency and availability in a partition-tolerant system
% - Eventual consistency: sacrifize consistency to ensure availability in case of network failure
% - Necessity of a conflict resolution mechanism

In order to design large scale distributed systems, the literature and companies increasingly adopt the optimistic replication model known as eventual consistency to replicate data among nodes.
This consistency model allows replicas to temporarily diverge to be able to ensure high availability.
Each node owning a copy of the data can edit it without any kind of coordination with other nodes, before propagating the changes to others.
A conflict resolution mechanism is however required to handle updates generated in parallel by different replicas.

% CRDT
% - Mimic behavior of traditional data structures
% - Designed for a distributed usage
% - Used in distributed storages and collaborative editing tools
% - Comes in several flavors: State-based and Operation-based

An approach which gains in popularity since a few years proposes to define \acfp{CRDT}\cite{shapiro:inria-00555588}.
These data structures behave as traditional ones, like the \emph{Set} or the \emph{Sequence} data structures, but are designed for a distributed usage.
Their specification ensures that concurrent changes are resolved deterministically and that replicas eventually converge after observing all updates.
Several promising ongoing works rely on this approach
to design distributed datastores \emph{TODO: insert reference to Riak or Antidote},
collaborative editing tools \cite{nicolas:hal-01655438}\emph{TODO: insert reference to teletype}
and even programming languages for designing distributed application \emph{TODO: insert reference to LASP}.

% State-based
% - Define a monotonically increasing state
% - Define a idempotent and commutative merge function
% - Allow to share updates by sending own state to other replicas so they merge it
% - No requirements for the network layer
% - But bandwith-consuming

% Operation-based
% - Define a set of operations to update the data structure
% - Define a partial order between operations
% - Concurrent operations commute
% - Allow to share updates by sending only corresponding operations
% - But network layer has to deliver operation while complying to their partial order

% Constraints on identifiers
% - Rely usually on identifiers attached to elements
% - Several constraints may exist on identifiers: globally unique, belonging to a dense set
% - Implementations (literature ?) do not provide identifiers with fixed size: depend on number of nodes...

To achieve convergence, \acp{CRDT} proposed in the literature mostly rely on identifiers to reference updated elements.
To be globally unique, element identifiers often include the identifier of the node which generates them.
But, since node identifiers grow as new nodes join the system, element identifiers have to grow proportionally.
Furthermore, element identifiers have to comply to additional constraints according to the \ac{CRDT}, which may result in the acceleration of their growth.

% First, identifiers are required to be globally unique for the lifetime of the data structure.
% For example, they may also be used to represent the order between updates.
% In that case, since the number of updates should not be limited, the identifiers have to grow to continue to satisfy this constraint.

% Research problem and questions
% - Size of metadata grows until eventually exceed the size used by data itself
% - How can we provide more efficient identifiers while still complying to their set of constraints?
% - How can we provide renaming mechanisms reducing the size of identifiers at some point?

Hence, since the size of identifiers is not bounded, the size of metadata attached to each element increases over time.
It thus exceeds more and more the size of data itself.
This impedes the adoption of \acp{CRDT} since nodes have to broadcast and store metadata, causing the application's performances and efficiency to decrease over time.

The goal of this PhD is to address this issue by
\begin{enumerate*}
  \item proposing more efficient specifications of identifiers according to their set of constraints,
  \item proposing mechanisms to rename identifiers to reduce their size.
\end{enumerate*}

% Next steps
% - Look for a survey (or build one) on \ac{CRDT} identifiers
%   - Which constraints on identifiers are found in the literature?
%   - How are these constraints addressed?
% - Can we instead provide \acp{CRDT} not relying on identifiers?

\bibliographystyle{plain}
\bibliography{ref-selected}

\end{document}

