%%%%%%%%%%%%%%%%%%%%%%%%%%%%%%%%%%%%%%%
% Deedy - One Page Two Column Resume
% LaTeX Template
% Version 1.1 (30/4/2014)
%
% Original author:
% Debarghya Das (http://debarghyadas.com)
%
% Original repository:
% https://github.com/deedydas/Deedy-Resume
%
% IMPORTANT: THIS TEMPLATE NEEDS TO BE COMPILED WITH XeLaTeX
%
% This template uses several fonts not included with Windows/Linux by
% default. If you get compilation errors saying a font is missing, find the line
% on which the font is used and either change it to a font included with your
% operating system or comment the line out to use the default font.
%
%%%%%%%%%%%%%%%%%%%%%%%%%%%%%%%%%%%%%%
%
% TODO:
% 1. Integrate biber/bibtex for article citation under publications.
% 2. Figure out a smoother way for the document to flow onto the next page.
% 3. Add styling information for a "Projects/Hacks" section.
% 4. Add location/address information
% 5. Merge OpenFont and MacFonts as a single sty with options.
%
%%%%%%%%%%%%%%%%%%%%%%%%%%%%%%%%%%%%%%
%
% CHANGELOG:
% v1.1:
% 1. Fixed several compilation bugs with \renewcommand
% 2. Got Open-source fonts (Windows/Linux support)
% 3. Added Last Updated
% 4. Move Title styling into .sty
% 5. Commented .sty file.
%
%%%%%%%%%%%%%%%%%%%%%%%%%%%%%%%%%%%%%%%
%
% Known Issues:
% 1. Overflows onto second page if any column's contents are more than the
% vertical limit
% 2. Hacky space on the first bullet point on the second column.
%
%%%%%%%%%%%%%%%%%%%%%%%%%%%%%%%%%%%%%%

\setlength{\parindent}{4em}
\setlength{\parskip}{1em}

\documentclass[]{deedy-resume-openfont}

\usepackage{ragged2e}
\usepackage{acronym} % \ac[p], \acl[p], \acs[p], \acf[p]

% Acronyms
% --------
\acrodef{CRDT}[CRDT]{Conflict-free Replicated Data Type}
\acrodefplural{CRDT}[CRDTs]{Conflict-free Replicated Data Types}

\begin{document}

%%%%%%%%%%%%%%%%%%%%%%%%%%%%%%%%%%%%%%
%
%     LAST UPDATED DATE
%
%%%%%%%%%%%%%%%%%%%%%%%%%%%%%%%%%%%%%%
%\lastupdated

%%%%%%%%%%%%%%%%%%%%%%%%%%%%%%%%%%%%%%
%
%     CONTACT INFORMATION
%
%%%%%%%%%%%%%%%%%%%%%%%%%%%%%%%%%%%%%%
%\contact

%%%%%%%%%%%%%%%%%%%%%%%%%%%%%%%%%%%%%%
%
%     TITLE NAME
%
%%%%%%%%%%%%%%%%%%%%%%%%%%%%%%%%%%%%%%
\namesection{Matthieu}{NICOLAS}{ \href{mailto:matthieu.nicolas@inria.fr}{matthieu.nicolas@inria.fr} | +33 6 75 98 34 40 }

%%%%%%%%%%%%%%%%%%%%%%%%%%%%%%%%%%%%%%
%     EXPERIENCE
%%%%%%%%%%%%%%%%%%%%%%%%%%%%%%%%%%%%%%

\section{Career}

\runsubsection{PhD Student}
\descript{| UNIVERSITY OF LORRAINE, COAST TEAM}
\location{October 2017 - Today | Nancy, France}
\sectionsep

\hfill\begin{minipage}{\dimexpr\textwidth-0.5cm}
\descript{EFFICIENT (RE)NAMING IN CONFLICT-FREE REPLICATED DATA TYPES}

\acp{CRDT} are data structures behaving as traditional ones, like the \emph{Set} or the \emph{Sequence} data structures, but designed for a distributed usage.
They are used in order to build large scale distributed systems adopting the optimistic replication model known as eventual consistency to replicate data among nodes.
With this model, each node owning a copy of the data can edit it without any kind of coordination with other nodes.
They then propagate the updates to others.
The specification of \acp{CRDT} ensures that concurrent updates are resolved deterministically and that replicas eventually converge after observing all of them.
\\
To achieve convergence, \acp{CRDT} proposed in the literature mostly rely on identifiers to reference updated elements.
According to the kind of \ac{CRDT}, identifiers have to comply to several constraints (unicity, forming a dense set...).
\\
Because of these constraints, the identifiers size is often not bounded.
Therefore, the size of metadata attached to each element increases with the number of updates.
It thus exceeds more and more the size of data itself, decreasing the efficiency of the data structure over time.
\\
The goal of this PhD is to address this issue by
\begin{tightemize}
  \item Proposing more efficient specifications of identifiers according to their set of constraints,
  \item Proposing mechanisms to rename identifiers to reduce their size.
\end{tightemize}
\sectionsep
\end{minipage}

\runsubsection{Research \& Development Software Engineer}
\descript{| INRIA, COAST TEAM}
\location{September 2014 – September 2017 | Nancy, France}
\sectionsep

\hfill\begin{minipage}{\dimexpr\textwidth-0.5cm}
\descript{PROJECT OPENPAAS::NG}
The goal of this project is to design an open-source entreprise social network providing a suite of
peer-to-peer collaborative office applications.
The aim is to offer a reliable and free alternative to existing solutions such as Google Apps.
This project is a joint work with the team DaSciM (Data Science and Mining) from the
computer science laboratory from the Ecole Polytechnique, Linagora, XWiki SAS and Nexedi.
\\
In this project, the COAST team works on topics such as the interorganisational federation of
peer-to-peer systems and the securing of communications in this kind of collaboration.
Furthermore, the team provides its expertise
on eventually consistent data replication mechanisms in distributed systems.
\\
In order to validate them, these works have been integrated in \href{https://www.coedit.re}{\customboldlink{MUTE}},
the demonstration platform of the team.
\begin{tightemize}
\item Maintaining of \emph{LogootSplit}\cite{sp2-l2-9}\cite{sp2-l2-10} implementation
\item Study of the literature on Conflict-free Replicated Data Types and of their use cases.
\item Development and integration of an anti-entropy mechanism\cite{sp2-l2-2}
\end{tightemize}
\sectionsep
\end{minipage}

\hfill\begin{minipage}{\dimexpr\textwidth-0.5cm}
\location{Publications}
\renewcommand\refname{\vskip -20pt} % Couldn't get this working from the .cls file
\bibliographystyle{abbrv}
\bibliography{publications}
\nocite{*}
\end{minipage}

\hfill\begin{minipage}{\dimexpr\textwidth-0.5cm}
\descript{ADT INRIA PLM}
\href{http://people.irisa.fr/Martin.Quinson/Teaching/PLM/}{\customboldlink{The PLM}}
is an open-source programming exerciser.
Developed by Gérald Oster and Martin Quinson, this application proposes to students
to explore and to learn several concepts of the algorithmic through interactive and graphical exercises.
\\
The goal of this project was to enhance this tool in an experimental platform dedicated
to the teaching of computer science.
To achieve this, a usage data collecting mechanism was required in order to generate a dataset.
This dataset, made available to researchers, allows the realisation of research works on topics
such as the design of an automatic helping tool which tailors error messages to students.
A second objective of this project was to port the application, until then released as
a desktop Java software, into a web application to make it available to the greatest number.
\\
My works focused mainly on the realisation of that port.
This major change of paradigm introduced several issues which needed to be addressed.
\begin{tightemize}
    \item Implementation and testing of the usage data collecting mechanism
    \item Conception and integration of a distributed architecture ensuring the scalability of the application
    \item Isolation of the execution of student code using containers
    \item Deployment and monitoring of a multi-components application
\end{tightemize}
\sectionsep\xdef\tpd{\the\prevdepth}
\end{minipage}

\runsubsection{Intern}
\descript{| UNIVERSITY OF LORRAINE, COAST TEAM}
\location{April 2014 – August 2014 | Nancy, France}
\sectionsep

\hfill\begin{minipage}{\dimexpr\textwidth-0.5cm}
\descript{Design of a collaborative editing tool}
Coming from the research on collaborative editing, a new family of data replication and consistency maintenance algorithms
was recently formalised : \acfp{CRDT} approach.
This new family addresses several unresolved issues from other approaches, especially its scalability.
\\
The COAST team proposed a new algorithm from this family : \emph{LogootSplit}.
\\
In order to illustrate and highlight the work of the team on this new approach,
my task was to design and implement a new real time collaborative editing tool based on this algorithm.
\begin{tightemize}
\item Implementation of \emph{LogootSplit} as a library
\item Design and development of \href{https://www.coedit.re}{\customboldlink{MUTE}}, a real time collaborative editing web app using this library
\end{tightemize}
\sectionsep\xdef\tpd{\the\prevdepth}
\end{minipage}

\runsubsection{Intern}
\descript{| POLYTECHNIQUE MONTRÉAL}
\location{September 2009 – June 2011 | Montreal, Canada}
\sectionsep

\hfill\begin{minipage}{\dimexpr\textwidth-0.5cm}
\descript{Development of a tool to check the correctness of collaborative editing algorithms}
Existing collaborative editing tools relies mostly on a specific family of algorithms
to ensure the eventual consistency of copies : the operational transformation.
\\
Two consistency properties \emph{TP1} and \emph{TP2} are defined and allow to ensure the correctness of algorithms from this family.
\\
The goal of this internship was to develop a tool to automatically check the respect of these properties for a given algorithm.
\begin{tightemize}
\item Implementation of several algorithms from the operational transformation family
\item Development of the tool allowing to check if the algorithms ensure \emph{TP1} and \emph{TP2}
\end{tightemize}
\sectionsep\xdef\tpd{\the\prevdepth}
\end{minipage}

\section{EDUCATION}

\runsubsection{Engineering degree in Computer Science at TELECOM Nancy}
\descript{}
\location{Equivalent to a Master Degree in Computer Science}
\location{September 2011 – August 2014 | Nancy, France}
\sectionsep

%%%%%%%%%%%%%%%%%%%%%%%%%%%%%%%%%%%%%%
%     COMMUNICATION
%%%%%%%%%%%%%%%%%%%%%%%%%%%%%%%%%%%%%%

\section{Communication}

\runsubsection{Presentations about MUTE}
\descript{}
\vspace{5pt}
\begin{tabular}{cp{150mm}}
August 2017         & ECSCW 2017\\
December 2016       & HCERES Evaluation of the LORIA\\
                    & Inria Industry Meeting "New technologies to protect digital data and computer systems"\\
                    & Visit of a delegation of Technological University presidents from Mexico\\
November 2016       & Inria Industry Meeting "Interaction with digital objects and services"\\
October 2016        & Evaluation seminar of Inria teams working on "Distributed Systems and Middleware"\\
\end{tabular}
\sectionsep

\hfill
\end{document}
